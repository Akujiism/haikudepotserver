% -----------
% Copyright 2013, Andrew Lindesay
% Distributed under the terms of the MIT License.
% -----------

\section{Overview and General Information}

These documents relate to an ``application-server'' called ``Haiku Depot Server'' which vends an internet application that stores and allows interaction with software packages for the \href{http://www.haiku-os.org}{Haiku Operating System}.

The application server is able to communicate with remote repositories to retrieve ``Haiku Package Repository'' (.hpkr) files.  .hpkr files contain meta-data of the packages that are stored at that repository.  The application server is able to consume the .hpkr data and populate its own internal database with some of the meta-data that it contains.

The application-server then provides an HTTP-vended API as well as a web user-interface to the data.  The application server is also intended to interact with a desktop client also called ``Haiku Depot''.  The system augments the information acquired from the .hpkr data with additional detail such as;

\begin{itemize}
\item Screenshots
\item Icons
\item Comments
\end{itemize}

\subsection{License}

The license can be found in the file {\tt LICENSE.TXT} at the top level of the project source.  The license also applies to this documentation.

\subsection{Hosted Source Code}

The source code is hosted at;

\framebox{http://code.google.com/p/haiku-depot-web-app/}

\subsection{Prerequisites}
\label{prerequisites}

\begin{itemize}
\item \href{https://community.java.net/open-jdk}{Java} $\geqslant$ 1.6
\item \href{http://maven.apache.org}{Maven} $\geqslant$ 3.0.3
\item \href{http://www.postgres.org}{Postgres} database $\geqslant$ 9.1
\end{itemize}

On a debian host, these prerequisites can be installed with;

\begin{verbatim}
apt-get install default-jdk
apt-get install maven
apt-get install postgresql postgresql-client
\end{verbatim}

\subsubsection{Building on Linux}
\label{prerequisites-buildingonlinux}

The build system for a linux host requires the presence of RPM assembly tools.

\begin{tabular}{|l|l|}
\hline
RPM-based Linux & {\tt yum install rpm-build} \\
Debian-based Linux & {\tt apt-get install rpm} \\
\hline
\end{tabular}

\subsubsection{Basic Postgres Setup}

The setup discussed here is {\bf very simplistic} as it is not possible to envisage all of the possible environmental factors involved in a production deployment.  By this point, the Postgres database server is  installed on a UNIX-like computer system and is running as the system user {\tt postgres}.

To get the Postgres database server to listen on an internet socket, uncomment the {\tt postgresql.conf} file line;

\framebox{\tt listen\_address = `localhost'}

In order to get connections to localhost to take authentication via username and password, edit the table at the end of the {\tt pg\_hba.conf} file by modifying the ``METHOD'' column for rows pertaining to the localhost; change {\tt ident} to {\tt md5}.

The Postgres database server should then be restarted.

Create a new database user with the following command;

\framebox{\tt sudo -u postgres createuser -P -E haikudepotserver}

Now create the new database;

\framebox{\tt sudo -u postgres createdb -O haikudepotserver haikudepotserver}

You can check the login to the database works by opening a SQL terminal;

\framebox{\tt psql -h localhost -U haikudepotserver haikudepotserver}

The database schema objects will be automatically populated into the fresh database by the application-server as it launches.

